\documentclass[a4paper,11pt]{book}
%\documentclass[a4paper,twoside,11pt,titlepage]{book}
\usepackage{listings}
\usepackage[utf8]{inputenc}
\usepackage[spanish]{babel}

% \usepackage[style=list, number=none]{glossary} %
%\usepackage{titlesec}
%\usepackage{pailatino}

%\decimalpoint
\usepackage{dcolumn}
\newcolumntype{.}{D{.}{\esperiod}{-1}}
\makeatletter
%\addto\shorthandsspanish{\let\esperiod\es@period@code}
\makeatother


%\usepackage[chapter]{algorithm}
\RequirePackage{verbatim}
%\RequirePackage[Glenn]{fncychap}
\usepackage{fancyhdr}
\usepackage{graphicx}
\usepackage{afterpage}

\usepackage{longtable}

\usepackage[pdfborder={000}]{hyperref} %referencia

% ********************************************************************
% Re-usable information
% ********************************************************************
\newcommand{\myTitle}{NO-INVENTORY\xspace}
\newcommand{\myDegree}{Grado en Ingeniería Informática\xspace}
\newcommand{\myName}{César Hugo Bárzano Cruz\xspace}
\newcommand{\myProf}{Nombre Apllido1 Apellido2 (tutor1)\xspace}
\newcommand{\myOtherProf}{Nombre Apllido1 Apellido2 (tutor2)\xspace}
%\newcommand{\mySupervisor}{Put name here\xspace}
\newcommand{\myFaculty}{Escuela Técnica Superior de Ingenierías Informática y de
Telecomunicación\xspace}
\newcommand{\myFacultyShort}{E.T.S. de Ingenierías Informática y de
Telecomunicación\xspace}
\newcommand{\myDepartment}{Departamento de Arquitectura y Tecnología de los Computadores\xspace}
\newcommand{\myUni}{\protect{Universidad de Granada}\xspace}
\newcommand{\myLocation}{Granada\xspace}
\newcommand{\myTime}{\today\xspace}
\newcommand{\myVersion}{Version 0.1\xspace}


\hypersetup{
pdfauthor = {\myName hugobarzano@correo.ugr.es},
pdftitle = {\myTitle},
pdfsubject = {},
pdfkeywords = {Gestión, Almacén , Inventario, Eficiencia, Ahorro, Informes,Web, Android },
pdfcreator = {LaTeX con el paquete TEXmaker},
pdfproducer = {pdflatex}
}

%\hyphenation{}


%\usepackage{doxygen/doxygen}
%\usepackage{pdfpages}
\usepackage{url}
\usepackage{colortbl,longtable}
\usepackage[stable]{footmisc}
%\usepackage{index}

%\makeindex
%\usepackage[style=long, cols=2,border=plain,toc=true,number=none]{glossary}
% \makeglossary

% Definición de comandos que me son tiles:
%\renewcommand{\indexname}{Índice alfabético}
%\renewcommand{\glossaryname}{Glosario}

\pagestyle{fancy}
\fancyhf{}
\fancyhead[LO]{\leftmark}
\fancyhead[RE]{\rightmark}
\fancyhead[RO,LE]{\textbf{\thepage}}
\renewcommand{\chaptermark}[1]{\markboth{\textbf{#1}}{}}
\renewcommand{\sectionmark}[1]{\markright{\textbf{\thesection. #1}}}

\setlength{\headheight}{1.5\headheight}

\newcommand{\HRule}{\rule{\linewidth}{0.5mm}}
%Definimos los tipos teorema, ejemplo y definición podremos usar estos tipos
%simplemente poniendo \begin{teorema} \end{teorema} ...
\newtheorem{teorema}{Teorema}[chapter]
\newtheorem{ejemplo}{Ejemplo}[chapter]
\newtheorem{definicion}{Definición}[chapter]

\definecolor{gray97}{gray}{.97}
\definecolor{gray75}{gray}{.75}
\definecolor{gray45}{gray}{.45}
\definecolor{gray30}{gray}{.94}

\lstset{ frame=Ltb,
     framerule=0.5pt,
     aboveskip=0.5cm,
     framextopmargin=3pt,
     framexbottommargin=3pt,
     framexleftmargin=0.1cm,
     framesep=0pt,
     rulesep=.4pt,
     backgroundcolor=\color{gray97},
     rulesepcolor=\color{black},
     %
     stringstyle=\ttfamily,
     showstringspaces = false,
     basicstyle=\scriptsize\ttfamily,
     commentstyle=\color{gray45},
     keywordstyle=\bfseries,
     %
     numbers=left,
     numbersep=6pt,
     numberstyle=\tiny,
     numberfirstline = false,
     breaklines=true,
   }

% minimizar fragmentado de listados
\lstnewenvironment{listing}[1][]
   {\lstset{#1}\pagebreak[0]}{\pagebreak[0]}

\lstdefinestyle{CodigoC}
   {
	basicstyle=\scriptsize,
	frame=single,
	language=C,
	numbers=left
   }
\lstdefinestyle{CodigoC++}
   {
	basicstyle=\small,
	frame=single,
	backgroundcolor=\color{gray30},
	language=C++,
	numbers=left
   }


\lstdefinestyle{Consola}
   {basicstyle=\scriptsize\bf\ttfamily,
    backgroundcolor=\color{gray30},
    frame=single,
    numbers=none
   }


\newcommand{\bigrule}{\titlerule[0.5mm]}


%Para conseguir que en las páginas en blanco no ponga cabecerass
\makeatletter
\def\clearpage{%
  \ifvmode
    \ifnum \@dbltopnum =\m@ne
      \ifdim \pagetotal <\topskip
        \hbox{}
      \fi
    \fi
  \fi
  \newpage
  \thispagestyle{empty}
  \write\m@ne{}
  \vbox{}
  \penalty -\@Mi
}
\makeatother

\usepackage{pdfpages}
\begin{document}
\begin{titlepage}
 
 
\newlength{\centeroffset}
\setlength{\centeroffset}{-0.5\oddsidemargin}
\addtolength{\centeroffset}{0.5\evensidemargin}
\thispagestyle{empty}

\noindent\hspace*{\centeroffset}\begin{minipage}{\textwidth}

\centering
\includegraphics[width=0.9\textwidth]{imagenes/logo_ugr.jpg}\\[1.4cm]

%\textsc{ \Large CERTAMEN DE PROYECTOS\\[0.2cm]}
% Upper part of the page
% 
% Title
{\Huge\bfseries CERTAMEN DE PROYECTOS\\
}
\noindent\rule[-1ex]{\textwidth}{3pt}\\[3.5ex]
{\large\bfseries NO-INVENTORY: Sistema de Gestión para Inventarios}
\end{minipage}

\vspace{2.5cm}
\noindent\hspace*{\centeroffset}\begin{minipage}{\textwidth}
\centering

\textbf{Autor}\\ {César Hugo Bárzano Cruz}\\[2.5ex]

\includegraphics[width=0.3\textwidth]{imagenes/etsiit_logo.png}\\[0.1cm]
\textsc{Escuela Técnica Superior de Ingenierías Informática y de Telecomunicación}\\
\textsc{---}\\
Granada, Mayo de 2016
\end{minipage}
%\addtolength{\textwidth}{\centeroffset}
%\vspace{\stretch{2}}
\end{titlepage}



%\chapter*{}
%\thispagestyle{empty}
%\cleardoublepage

%\thispagestyle{empty}

%\input{portada/portada_2}



\cleardoublepage
\thispagestyle{empty}

\begin{center}
{\large\bfseries NO-INVENTORY: Sistema de Gestión Para Almacenes}\\
\end{center}
\begin{center}
César Hugo Bárzano Cruz\\
\end{center}

%\vspace{0.7cm}
\noindent{\textbf{Palabras clave}: Gestión, Almacén,Inventario, Activos, Eficiencia, Ahorro, Catalogación, Clasificación, Informes, Gráficos, Móvil, Código Abierto, Servicio Nube}\\

\vspace{0.7cm}
\noindent{\textbf{Resumen}}\\

Este proyecto surge a raíz de un problema real presentado por la Oficina de Software Libre. Desde principios de 2012 y a partir de un acuerdo con la Unidad de Calidad, la oficina se encarga de recoger material informático procedente de los distintivos organismos de la Universidad de Granada. Dicho material ha alcanzado una gran cantidad, por lo que es necesario un sistema para gestionarlo. 
Partiendo de este problema, existen necesidades similares en empresas, organizaciones, instituciones, comercios y todo tipo de negocio con un almacén de activos.

En dichos almacenes, las tareas de gestión suelen ser ineficientes, complejas y costosas.  El tiempo y dinero que conllevan estas tareas suele ser un factor a tener en cuenta, ya que una gran mayoría de empresas, con almacenes pequeños o medianos, no utilizan un sistema comercial debido a los costes que suponen su mantenimiento. Este problema, sigue presente en organizaciones con almacenes de gran tamaño, ya que  el número de empleados y el tiempo necesario para las tareas básicas de catalogación y administración suponen un coste a tener en cuenta.

Otro problema con el que se encuentran los empleados de estas entidades es la generación de informes. Recopilar información de los activos del almacén con el objetivo de representarla de manera adecuada a las necesidades de cada negocio, puede llegar a ser una tarea complicada. 
Como solución a estos problemas, surge el sistema de gestión NO-INVENTORY que pretende mejorar estas tareas, con el objetivo de ahorrar tiempo y dinero al cliente, automatizando y facilitando las tareas de gestión. 
La piedra angular del sistema es una plataforma web alojada en la nube por lo que no es necesaria la instalación de ningún software adicional en las máquinas del cliente que deseé comenzar a utilizarlo, solo es necesario un navegador.  En dicha plataforma, las tareas de administración se realizan de manera intuitiva permitiendo a los distintos empleados trabajar de forma cooperativa dentro del entorno colaborativo que representa al almacén de su empresa, utilizando un sistema flexible y personalizable en función de las características con las que se quieran clasificar los elementos. Facilita la agrupación de elementos con o sin propiedades comunes en colecciones denominadas catálogos en función de las necesidades de cada cliente y utiliza dichas colecciones para la generación automática de informes y gráficos representativos del estado del almacén. En función de la cantidad de datos o de la integridad de los mismo, la infraestructura subyacente del proyecto, permitiría correr instancias aisladas del sistema, para dar servicio solo a esa empresa, aprovechando así los recursos de la plataforma al máximo. 

El sistema cuenta con una aplicación Android como extensión para realizar tareas de catalogación y clasificación dentro del propio almacén. La ventaja de esto reside en que hoy en día casi todo el mundo cuenta con un smartphone, dando la posibilidad a los empleados de llevar la gestión del almacén con sus propios dispositivos.

En función de los activos que formen el almacén, y del presupuesto que se quiera dedicar a etiquetar y clasificar cada objeto, el cliente puede decidir que método utilizar, ya que la funcionalidad de la aplicación Android es la de leer y escribir los identificadores de cada elemento, con soporte para:
\begin{enumerate}
\item Códigos de Barras
\item Códigos QR
\item Etiquetas NFC 
\end{enumerate}

En resumen, el sistema pretende reducir tiempo y dinero a las organizaciones que decidan utilizarlo para la gestión de sus almacenes con el objetivo  de optimizar las tareas de los empleados consiguiendo así un mayor rendimiento. 
Por último, resaltar que todo lo relativo al desarrollo del proyecto, tanto código como documentación, está liberado en el sistema de control de versiones Github, bajo una licencia GPL3 para que pueda ser utilizado o mejorado por la comunidad de software libre.


\cleardoublepage


\thispagestyle{empty}


\begin{center}
{\large\bfseries NO-INVENTORY: Warehouse Management System}\\
\end{center}
\begin{center}
Bárzano Cruz, César Hugo\\
\end{center}

%\vspace{0.7cm}
\noindent{\textbf{Keywords}: Management, Warehouse, Inventory, Assets, Efficiency, Savings, Cataloging, Classification, Reports, Graphics, Smartphone, Open Source, Cloud Service}\\

\vspace{0.7cm}
\noindent{\textbf{Abstract}}\\

This project is in origin born because of a real problem being faced by the 'Oficina de Software Libre' since early 2012 and partly due to an arrangement made with the 'Unidad de Calidad', which is the entity that collects all the computer equipment coming from different departments of the University of Granada. The equipment assets have really being increased so a new need of a management system appears. Taking this as starting point, similar need appears in private companies, additional public organisms, trades and any kind of business with a warehouse in between.

In these warehouses, the management tasks are in general inefficient, complex and expensive.
Time and budget needed for the maintenance tasks are the main factors that make small and medium sized companies not considering the idea of using a commercial management system. This two factors are also present in large-sized organizations with bigger installations and capabilities because the number of employees and the time needed to perform the categorization and administration tasks are again a big cost to assume.


Another problem that employees face in their day to day activities is the generation of meaningful reports. Gathering information of the ware assets with the objective of presenting this in a proper format adequate to the requirements of each business can be very complicated. As a solution proposal for all these warehouses issues, we have created NO-INVENTORY management system that has been designed to improve all these tasks, with the main objective of saving money and time to the final customer by automating and facilitating all the management tasks.

The cornerstone of this management system is that it is a web based tool located in the cloud so there is no need of installation of any additional software on final user side. The only requirement to start using it is a regular web explorer. In this platform, management tasks are driven in an intuitive way allowing different final users to cooperate within a collaborative environment that represents the warehouse of their company, using a flexible customizable system dependent on the specific characteristics wanted for the elements classification. It facilitates the grouping of elements with or without common properties in collections called catalogs based on the needs of every customer and uses those collections for the automated generation of reports, graphics that help to represent the current state of the warehouse. In function of the the amount of data or the data integrity, the infrastructure behind the project, will allow running isolated instances, to provide service to that company, maximizing this way the resources of the platform to the maximum.


The system includes, as extension, an Android application, to perform the classification and categorization and tasks inside the actual warehouse. The advantage of this is that nowadays almost everyone owns a smart-phone, so final users can carry out the warehouse management via their own device. Dependant of the assets belonging to the warehouse and the budget agreed for labeling and classifying each object, the customer can decide which method to use, as the android application will be used for reading and writing the IDs of each element, and gives support to the next methods:


\begin{enumerate}
\item Barcodes
\item QRcodes
\item NFC tags
\end{enumerate}

As a final summary, this system helps to reduce time and money to organizations that decide to use it for the management of their warehouses by introducing and automated solid and defined process that will increase the performance and will optimize the daily challenges the employees have to deal with. Finally, note that all matters relating to the development of the project, both code and documentation, is released into the version control system GitHub under a GPL3 license so it can be used or improved by the free software community. 


\chapter*{}
\thispagestyle{empty}

\noindent\rule[-1ex]{\textwidth}{2pt}\\[4.5ex]

Yo, \textbf{César Hugo Bárzano Cruz}, alumno de la titulación Grado en Ingeniería Informática de la \textbf{Escuela Técnica Superior de Ingenierías Informática y de Telecomunicación de la Universidad de Granada}, con DNI 77138361h, autorizo la ubicación de la siguiente copia de mi Trabajo Fin de Grado en la biblioteca del centro para que pueda ser consultada por las personas que lo deseen.

\vspace{6cm}

\noindent Fdo: César Hugo Bárzano Cruz

\vspace{2cm}

\begin{flushright}
Granada a 9 de Julio de 2016 .
\end{flushright}


\chapter*{}
\thispagestyle{empty}

\noindent\rule[-1ex]{\textwidth}{2pt}\\[4.5ex]

D. \textbf{Juan Julián Merelo Guervós }, Profesor del Área de ATC del Departamento Arquitectura y Tecnología de Computadores de la Universidad de Granada.

\vspace{0.5cm}


\vspace{0.5cm}

\textbf{Informa:}

\vspace{0.5cm}

Que el presente trabajo, titulado \textit{\textbf{NO-INVENTORY, Sistema de Gestión Para Almacenes}},
ha sido realizado bajo su supervisión por \textbf{César Hugo Bárzano Cruz}, y autoriza la defensa de dicho trabajo ante el tribunal que corresponda.

\vspace{0.5cm}

Y para que conste, expide y firma el presente informe en Granada a 9 de Julio de 2016 .

\vspace{1cm}

\textbf{El director:}

\vspace{5cm}

\noindent \textbf{Juan Julián Merelo Guervós \ \ \ \ \ }

\chapter*{Agradecimientos}
\thispagestyle{empty}

       \vspace{1cm}


A mi tutor, a mis compañeros y a mis queridas hermanas. 


%\frontmatter
\tableofcontents
%\listoffigures
%\listoftables

%
%\mainmatter
%\setlength{\parskip}{5pt}

%\input{capitulos/01_Introduccion}


\chapter{Introducción}


Este proyecto tiene el objetivo de solucionar un problema real. Desde principios de 2012, la Oficina de Software Libre (OSL) tiene un acuerdo con Unidad de Calidad de la Universidad de Granada. Dicho acuerdo establece que la oficina es la encargada de recoger el material  informático procedente de los distintivos organismos de la Universidad. Debido a la acumulación de material a lo largo de estos 5 años, la oficina de software libre necesita un sistema de gestión que le permita catalogar y clasificar este material.

El problema principal radica en el desconocimiento del estado en el que este material es recogido, es decir, la OSL no sabe si el material recibido funciona, no funciona, necesita ser actualizado o directamente es inservible y ha de ser reciclado. Hay que tener en cuenta el estado actual del almacén. Tras acumular material a lo largo de estos años, el almacén sufre una situación caótica debido al gran numero de equipos sin catalogar ni reflejar el estado en el que se encuentran. 

Por otra parte, el acuerdo mencionado anteriormente, obliga a la oficina a realizar informes cada cierto periodo de tiempo, reflejando el estado de las campañas de recogida, el estado del material recolectado, el peso...etc

El problema presentado no es algo aislado a la OSL, gran parte de las pequeñas y medianas empresas con almacenes que gestionar presentan problemas similares. Para las  empresas, no usar plataformas de gestión para sus productos supone carencia de estrategias de crecimiento, una inadecuada utilización de las tecnologías y conocimientos, pérdidas de recursos, debilidad financiera y deficiencias en toda la organización. 

En los últimos años, el análisis de la información asociada a los recursos o servicios de las empresas, ha tomado una gran importancia. El análisis de la información se ha convertido en un factor clave para la toma de decisiones en diversos sectores. El valor de la información ha alcanzado el nivel de otros recursos como por ejemplo los recursos financieros, materiales y humanos. El conocimiento del entorno y la información que podemos aprovechar de él son vitales para la toma de decisiones, ya sea fuera del entorno empresarial o dentro de él. Decisiones como por ejemplo entrar en nuevos mercados, proteger la empresa de agentes externos, innovación, desarrollo, inversiones...etc. Por ello las Tecnologías de Información forman un factor decisivo para dar lugar al crecimiento tanto de las PyMEs como de cualquier empresa.

Lo principales motivos por los que estas empresas no utilizan sistemas de gestión para sus productos se deben a los costes elevados, a la carencia de recursos, la falta de acceso a la información, etc... 

Las Tecnologías de la Información y Comunicaciones (TIC), se han convertido en la herramienta perfecta para llevar a cabo las tareas necesarias para aumentar 
la eficacia y eficiencia en el manejo de los recursos de la empresa. 


\chapter{Objetivos}

Objetivo general a conseguir: se enuncia lo que se quiere hacer sin entrar en detalles.

Objetivos específicos: Es dividir el objetivo general en los pasos a seguir con sub-objetivos más simples.
Dicho de otro modo, son cada uno de los pasos a realizar para alcanzar el objetivo general, es decir solucionando todos y cada uno de los objetivos específicos se resuelve el objetivo general. (Poner entre 3 y 5 como mucho).

Se puede decir en qué apartado se tratará cada objetivo específico.

Poner también los aspectos formativos previos utilizados, por ejemplo si se han usado técnicas de visión concretas como Transformada de Hough, Método de detección de rostros Viola-Jones, Filtro de Partículas, o técnicas de aprendizaje por SVM, etc. Se explica un poco cada método.

\chapter{Resolución del trabajo}

Como método de ingeniería del software decir que vamos a seguir la técnicas de modelo de prototipos rápido o también llamado modelado de prototipado rápido.

Ver \url{http://www.ecured.cu/index.php/Modelo_de_Prototipos}

\section{Recursos}

Decir los recursos humanos (autor y directores), hardware y software que se van a utilizar.

\section{Especificación de requisitos}

Decir que se partió de una especificación inicial de requisitos que a medida que se fueron implementando los prototipos se fue refinando posteriormente. Se puede poner la inicial y la final o solo la final indicando que se están poniendo los requisitos que finalmente tiene que tener el sistema.

Los requisitos se pueden referir a las necesidades del usuario del sistema (requisitos del usuario), a lo que tiene que hacer la aplicación (requisito funcional) o a cómo tiene que hacerlo (requisito no funcional). Ejemplo:

En este sistema un robot tiene que coger con sus pinzas un envase de medicamento y llevárselo a una persona anciana que por sí misma no puede recordar su medicación.

Requisitos del usuario:

RU1. La persona puede moverse libremente por una habitación donde está el robot.

RU2. La persona es capaz de coger el medicamento cuando se lo ofrece el robot.

RU3. La persona es capaz de tomarse el medicamento por sí misma.


Requisitos funcionales.

RF1. El robot mediante la cámara kinect debe poder localizar a la persona.

RF2. El robot conoce la posición de la mesa pues tiene un mapa de la habitación.

RF3. El robot debe identificar el medicamento correcto según un plan de medicación previamente establecido.

RF4. El robot debe poder coger el medicamento con sus pinzas.

etc...

Requisitos no funcionales

RNF1. El robot no puede atropellar ni dañar a la persona en ningún momento.

RNF2. La aplicación debe ejecutarse en entornos linux

RNF3. La aplicación debe utilizar pocos recursos para reaccionar con rapidez.

algo de la interfaz, como tratar posibles fallos, etc.


\section{Planificación}

Poner una tabla de tiempos con las planificación del proyecto diciendo cuando se tiene previsto alcanzar cada subobjetivo planteado. Con su correspondiente división en fases y tareas, y la posterior
comparación con los datos reales obtenidos tras realizar el proyecto. Entre las fases está la realización de los diferentes prototipos I, II y III por ejemplo.

Poner presupuesto según horas de trabajo estimadas.


\section{Análisis funcional}

A partir de aquí nos referimos solamente al prototipo final que da lugar a la aplicación final.

Hay que describir la funcionalidad que debe poseer el sistema para poder cumplir con los objetivos y requisitos que se han dicho previamente.La descripción de esta funcionalidad puede hacerse analizando las tareas (que aparecerán en la planificación) y estudiando la inter-relación entre ellas y sus conexiones.

Para la realización de este análisis se pueden utilizar Diagramas de Flujo para poder conocer generalmente un único punto de inicio y un único punto de término o en varios. 

\url{https://es.wikipedia.org/wiki/Diagrama_de_flujo}

 Se pueden plantear también casos de uso. Los diagramas de casos de uso sirven para describir la inter-relación entre el sistema y el usuario del mismo. Se pueden utilizar para plantear diferentes casos de interacción entre el robot y la persona y cómo tiene que reaccionar el sistema en cada caso.

\url{https://es.wikipedia.org/wiki/Diagrama_de_casos_de_uso}


\section{Implementación y pruebas}

Decir qué lenguaje de programación se ha utilizado y las tecnologías implicadas aunque se hayan comentado en el apartado de recursos. Justificar su uso (rendimiento, disponibilidad, etc.). Si se ha usado open source decirlo y explicar las ventajas.

Las pruebas se realizan para comprobar la verificación y validación del producto software. La verificación consiste es comprobar que el producto realiza lo que está programado, es decir la programación no tiene errores y funciona en todos los casos cumpliendo los requisitos. La validación tiene que ver con que cumpla con lo que espera el usuario.

Verificación y Validación: Conjunto de procesos de comprobación y
análisis que aseguran que el software que se desarrolla está acorde a su
especificación y cumple las necesidades de los clientes.

Verificación:
¿Estamos construyendo el producto correctamente?
e su especificacin.

Validación:
¿Estamos construyendo el producto correcto?
Comprueba que el software cumple las expectativas que el cliente espera
Importante: Nunca se va a poder demostrar que el software está
completamente libre de defectos

las pruebas que pueden utilizarse son muy diversas. Aconsejo centrarnos en pruebas de caja blanca y de caja negra.

Las pruebas de la caja blanca se centran en la estructura interna del programa para elegir los casos de prueba. El objetivo de estas pruebas consiste en probar todos los posibles casos de ejecución de la aplicación para comprobar que los datos se comportan de manera correcta internamente.

Decir que se han hecho las pruebas de caja blanca.

Las pruebas de caja negra son aquellas que se centran en las salidas y entradas de los módulos, sin atender a su comportamiento interno (comprobando mediante las pruebas de caja blanca). Las pruebas de caja negra garantizan la interconectividad entre los diferentes módulos de la aplicación, así como su correcto funcionamiento final.

Poner algunos casos de prueba de caja negra.

\chapter{Conclusiones y trabajo futuro}

Decir lo que se ha conseguido realizar comentando sus puntos fuertes y débiles.
Decir si se han alcanzado los objetivos específicos y el general propuesto y en qué grado.

Indicar las asignaturas del grado más relacionadas con la ejecución del TFG y cómo el TFG ha ayudado a afianzar los conocimientos adquiridos en el Grado.

Valoración personal si se quiere.


Avanzar algunas líneas de trabajo futuro para solucionar las debilidades detectadas o para conseguir nuevas funcionalidades interesantes.

\chapter{Bibliografía}

Poner las citas bibliográficas, direcciones de internet, etc.


\chapter{Anexo}

Al final de la memoria hay que añadir un anexo de una página o dos, explicando como se usa el software a modo de manual de usuario. Es decir, como se llaman los comandos, qué parámetros hay que darle, como se llaman los ficheros de datos de entrada, etc.



%
%\input{capitulos/02_EspecificacionRequisitos}
%
%\input{capitulos/03_Planificacion}
%
%\input{capitulos/04_Analisis}
%
%\input{capitulos/05_Diseno}
%
%\input{capitulos/06_Implementacion}
%
%\input{capitulos/07_Pruebas}
%
%\input{capitulos/08_Conclusiones}
%
%%\chapter{Conclusiones y Trabajos Futuros}
%
%
%%\nocite{*}
%\bibliography{bibliografia/bibliografia}\addcontentsline{toc}{chapter}{Bibliografía}
%\bibliographystyle{miunsrturl}
%
%\appendix
%\input{apendices/manual_usuario/manual_usuario}
%%\input{apendices/paper/paper}
%\input{glosario/entradas_glosario}
% \addcontentsline{toc}{chapter}{Glosario}
% \printglossary
\chapter*{}
\thispagestyle{empty}

\end{document}
