\chapter*{}
%\thispagestyle{empty}
%\cleardoublepage

%\thispagestyle{empty}

\input{portada/portada_2}



\cleardoublepage
\thispagestyle{empty}

\begin{center}
{\large\bfseries NO-INVENTORY: Sistema de gestión para inventarios}\\
\end{center}
\begin{center}
César Hugo Bárzano Cruz\\
\end{center}

%\vspace{0.7cm}
Gestión, Almacén , Activos, Eficiencia, Ahorro, Informes, 
\noindent{\textbf{Palabras clave}: Gestión\_clave1, Almacen\_clave2, Activos\_clave3, Eficiencia\_clave4, Ahorro\_clave5 Informes \_clave6}\\

\vspace{0.7cm}
\noindent{\textbf{Resumen}}\\

Este proyecto surge a raíz de un problema real presentado por la Oficina de Software Libre. Desde principios de 2012 y a partir de un acuerdo con la Unidad de Calidad, la oficina se encarga de recoger material informático procedente de los distintivos organismos de la Universidad de Granada. Dicho material ha alcanzado una gran cantidad, por lo que es necesario un sistema para gestionarlo. 
Partiendo de este problema, existen necesidades similares en empresas, organizaciones, instituciones, comercios y todo tipo de negocio con un almacén de activos.

En dichos almacenes, las tareas de gestión suelen ser ineficientes, complejas y costosas.  El tiempo y dinero que conllevan estas tareas suele ser un factor a tener en cuenta, ya que una gran mayoría de empresas, con almacenes pequeños o medianos, no utilizan un sistema comercial debido a los costes que suponen su mantenimiento. Este problema, sigue presente en organizaciones con almacenes de gran tamaño, ya que  el número de empleados y el tiempo necesario para las tareas básicas de catalogación y administración suponen un coste a tener en cuenta.

Otro problema con el que se encuentran los empleados de estas entidades es la generación de informes. Recopilar información de los activos del almacén con el objetivo de representarla de manera adecuada a las necesidades de cada negocio, puede llegar a ser una tarea complicada. 
Como solución a estos problemas, surge el sistema de gestión NO-INVENTORY que pretende mejorar estas tareas, con el objetivo de ahorrar tiempo y dinero al cliente, automatizando y facilitando las tareas de gestión. 
La piedra angular del sistema es una plataforma web alojada en la nube por lo que no es necesaria la instalación de ningún software adicional en las máquinas del cliente que deseé comenzar a utilizarlo, solo es necesario un navegador.  En dicha plataforma, las tareas de administración se realizan de manera intuitiva permitiendo a los distintos empleados trabajar de forma cooperativa dentro del entorno colaborativo que representa al almacén de su empresa, utilizando un sistema flexible y personalizable en función de las características con las que se quieran clasificar los elementos. Facilita la agrupación de elementos con o sin propiedades comunes en colecciones denominadas catálogos en función de las necesidades de cada cliente y utiliza dichas colecciones para la generación automática de informes y gráficos representativos del estado del almacén. En función de la cantidad de datos o de la integridad de los mismo, la infraestructura subyacente del proyecto, permitiría correr instancias aisladas del sistema, para dar servicio solo ha esa empresa, aprovechando así los recursos de la plataforma al máximo. 

El sistema cuenta con una aplicación Android como extensión para realizar tareas de catalogación y clasificación dentro del propio almacén. La ventaja de esto reside en que hoy en día casi todo el mundo cuenta con un smartphone, dando la posibilidad a los empleados de llevar la gestión del almacén con sus propios dispositivos.

En función de los activos que formen el almacén, y del presupuesto que se quiera dedicar a etiquetar y clasificar cada objeto, el cliente puede decidir que método utilizar, ya que la funcionalidad de la aplicación Android es la de leer y escribir los identificadores de cada elemento, con soporte para:
\begin{enumerate}
\item Códigos de Barras
\item Códigos QR
\item Etiquetas NFC 
\end{enumerate}

En resumen, el sistema pretende reducir tiempo y dinero a las organizaciones que decidan utilizarlo para la gestión de sus almacenes con el objetivo  de optimizar las tareas de los empleados consiguiendo así un mayor rendimiento. 
Por último, resaltar que todo lo relativo al desarrollo del proyecto, tanto código como documentación, está liberado en el sistema de control de versiones Github, bajo una licencia GPL3 para que pueda ser utilizado o mejorado por la comunidad de software libre.


\cleardoublepage


\thispagestyle{empty}


\begin{center}
{\large\bfseries Project Title: Project Subtitle}\\
\end{center}
\begin{center}
First name, Family name (student)\\
\end{center}

%\vspace{0.7cm}
\noindent{\textbf{Keywords}: Keyword1, Keyword2, Keyword3, ....}\\

\vspace{0.7cm}
\noindent{\textbf{Abstract}}\\

Write here the abstract in English.

\chapter*{}
\thispagestyle{empty}

\noindent\rule[-1ex]{\textwidth}{2pt}\\[4.5ex]

Yo, \textbf{César Hugo Bárzano Cruz}, alumno de la titulación TITULACIÓN de la \textbf{Escuela Técnica Superior
de Ingenierías Informática y de Telecomunicación de la Universidad de Granada}, con DNI 77138361h, autorizo la
ubicación de la siguiente copia de mi Trabajo Fin de Grado en la biblioteca del centro para que pueda ser
consultada por las personas que lo deseen.

\vspace{6cm}

\noindent Fdo: Nombre Apellido1 Apellido2

\vspace{2cm}

\begin{flushright}
Granada a X de mes de 201 .
\end{flushright}


\chapter*{}
\thispagestyle{empty}

\noindent\rule[-1ex]{\textwidth}{2pt}\\[4.5ex]

D. \textbf{Nombre Apellido1 Apellido2 (tutor1)}, Profesor del Área de XXXX del Departamento YYYY de la Universidad de Granada.

\vspace{0.5cm}

D. \textbf{Nombre Apellido1 Apellido2 (tutor2)}, Profesor del Área de XXXX del Departamento YYYY de la Universidad de Granada.


\vspace{0.5cm}

\textbf{Informan:}

\vspace{0.5cm}

Que el presente trabajo, titulado \textit{\textbf{Título del proyecto, Subtítulo del proyecto}},
ha sido realizado bajo su supervisión por \textbf{Nombre Apellido1 Apellido2 (alumno)}, y autorizamos la defensa de dicho trabajo ante el tribunal
que corresponda.

\vspace{0.5cm}

Y para que conste, expiden y firman el presente informe en Granada a X de mes de 201 .

\vspace{1cm}

\textbf{Los directores:}

\vspace{5cm}

\noindent \textbf{Nombre Apellido1 Apellido2 (tutor1) \ \ \ \ \ Nombre Apellido1 Apellido2 (tutor2)}

\chapter*{Agradecimientos}
\thispagestyle{empty}

       \vspace{1cm}


Poner aquí agradecimientos...

