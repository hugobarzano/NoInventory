\chapter*{}
%\thispagestyle{empty}
%\cleardoublepage

%\thispagestyle{empty}

\input{portada/portada_2}



\cleardoublepage
\thispagestyle{empty}

\begin{center}
{\large\bfseries NO-INVENTORY: Sistema de gestión para inventarios}\\
\end{center}
\begin{center}
César Hugo Bárzano Cruz\\
\end{center}

%\vspace{0.7cm}
Gestión, Almacén , Activos, Eficiencia, Ahorro, Informes, 
\noindent{\textbf{Palabras clave}: Gestión\_clave1, Almacen\_clave2, Activos\_clave3, Eficiencia\_clave4, Ahorro\_clave5 Informes \_clave6}\\

\vspace{0.7cm}
\noindent{\textbf{Resumen}}\\

Este proyecto surge a raíz de un problema real presentado por la Oficina de Software Libre. Desde principios de 2012 y a partir de un acuerdo con la Unidad de Calidad, la oficina se encarga de recoger material informático procedente de los distintivos organismos de la Universidad de Granada. Dicho material ha alcanzado una gran cantidad, por lo que es necesario un sistema para gestionarlo. 
Partiendo de este problema, existen necesidades similares en empresas, organizaciones, instituciones, comercios y todo tipo de negocio con un almacén de activos.

En dichos almacenes, las tareas de gestión suelen ser ineficientes, complejas y costosas.  El tiempo y dinero que conllevan estas tareas suele ser un factor a tener en cuenta, ya que una gran mayoría de empresas, con almacenes pequeños o medianos, no utilizan un sistema comercial debido a los costes que suponen su mantenimiento. Este problema, sigue presente en organizaciones con almacenes de gran tamaño, ya que  el número de empleados y el tiempo necesario para las tareas básicas de catalogación y administración suponen un coste a tener en cuenta.

Otro problema con el que se encuentran los empleados de estas entidades es la generación de informes. Recopilar información de los activos del almacén con el objetivo de representarla de manera adecuada a las necesidades de cada negocio, puede llegar a ser una tarea complicada. 
Como solución a estos problemas, surge el sistema de gestión NO-INVENTORY que pretende mejorar estas tareas, con el objetivo de ahorrar tiempo y dinero al cliente, automatizando y facilitando las tareas de gestión. 
La piedra angular del sistema es una plataforma web alojada en la nube por lo que no es necesaria la instalación de ningún software adicional en las máquinas del cliente que deseé comenzar a utilizarlo, solo es necesario un navegador.  En dicha plataforma, las tareas de administración se realizan de manera intuitiva permitiendo a los distintos empleados trabajar de forma cooperativa dentro del entorno colaborativo que representa al almacén de su empresa, utilizando un sistema flexible y personalizable en función de las características con las que se quieran clasificar los elementos. Facilita la agrupación de elementos con o sin propiedades comunes en colecciones denominadas catálogos en función de las necesidades de cada cliente y utiliza dichas colecciones para la generación automática de informes y gráficos representativos del estado del almacén. En función de la cantidad de datos o de la integridad de los mismo, la infraestructura subyacente del proyecto, permitiría correr instancias aisladas del sistema, para dar servicio solo a esa empresa, aprovechando así los recursos de la plataforma al máximo. 

El sistema cuenta con una aplicación Android como extensión para realizar tareas de catalogación y clasificación dentro del propio almacén. La ventaja de esto reside en que hoy en día casi todo el mundo cuenta con un smartphone, dando la posibilidad a los empleados de llevar la gestión del almacén con sus propios dispositivos.

En función de los activos que formen el almacén, y del presupuesto que se quiera dedicar a etiquetar y clasificar cada objeto, el cliente puede decidir que método utilizar, ya que la funcionalidad de la aplicación Android es la de leer y escribir los identificadores de cada elemento, con soporte para:
\begin{enumerate}
\item Códigos de Barras
\item Códigos QR
\item Etiquetas NFC 
\end{enumerate}

En resumen, el sistema pretende reducir tiempo y dinero a las organizaciones que decidan utilizarlo para la gestión de sus almacenes con el objetivo  de optimizar las tareas de los empleados consiguiendo así un mayor rendimiento. 
Por último, resaltar que todo lo relativo al desarrollo del proyecto, tanto código como documentación, está liberado en el sistema de control de versiones Github, bajo una licencia GPL3 para que pueda ser utilizado o mejorado por la comunidad de software libre.


\cleardoublepage


\thispagestyle{empty}


\begin{center}
{\large\bfseries Project Title: Project Subtitle}\\
\end{center}
\begin{center}
First name, Family name (student)\\
\end{center}

%\vspace{0.7cm}
\noindent{\textbf{Keywords}: Keyword1, Keyword2, Keyword3, ....}\\

\vspace{0.7cm}
\noindent{\textbf{Abstract}}\\

This project is in origin born because of a real problem being faced by the 'Oficina de Software Libre' since early 2012 and partly due to an arrangement made with the 'Unidad de Calidad', which is the entity that collects all the computer equipment coming from different departments of the University of Granada. The equipment assets have really being increased so a new need of a management system appears.

Taking this as starting point, similar need appears in private companies, additional public organisms, trades and any kind of business with a warehouse in between.


In these warehouses, the management tasks are in general inefficient, complex and expensive.

Time and budget needed for the maintenance tasks are the main factors that make small and medium sized companies not considering the idea of using a commercial management system.


This two factors are also present in large-sized organizations with bigger installations and capabilities because the number of employees and the time needed to perform the categorization and administration tasks are again a big cost to assume.


Another problem that employees face in their day to day activities is the generation of meaningful reports. Gathering information of the ware assets with the objective of presenting this in a proper format adequate to the requirements of each business can be very complicated.


As a solution proposal for all these warehouses issues, we have created NO-INVENTORY management system that has been designed to improve all these tasks, with the main objective of saving money and time to the final customer by automating and facilitating all the management tasks.


The cornerstone of this management system is that it is a web based tool located in the cloud so there is no need of installation of any additional software on final user side. The only requirement to start using it is a regular web explorer. In this platform, management tasks are driven in an intuitive way allowing different final users to cooperate within a collaborative environment that represents the warehouse of their company, using a flexible customizable system dependent on the specific characteristics wanted for the elements classification. It facilitates the grouping of elements with or without common properties in collections called catalogs based on the needs of every customer and uses those collections for the automated generation of reports, graphics that help to represent the current state of the warehouse. In function of the the amount of data or the data integrity, the infrastructure behind the project, will allow running isolated instances, to provide service to that company, maximizing this way the resources of the platform to the maximum.


The system includes, as extension, an Android application, to perform the classification and categorization and tasks inside the actual warehouse. The advantage of this is that nowadays almost everyone owns a smart-phone, so final users can carry out the warehouse management via their own device.

Dependant of the assets belonging to the warehouse and the budget agreed for labeling and classifying each object, the customer can decide which method to use, as the android application will be used for reading and writing the IDs of each element, and gives support to the next methods:


\begin{enumerate}
\item Barcodes
\item QRcodes
\item NFC tags
\end{enumerate}

As a final summary, this system helps to reduce time and money to organizations that decide to use it for the management of their warehouses by introducing and automated solid and defined process that will increase the performance and will optimize the daily challenges the employees have to deal with. 

Finally, note that all matters relating to the development of the project, both code and documentation, is released into the version control system GitHub under a GPL3 license so it can be used or improved by the free software community. 


\chapter*{}
\thispagestyle{empty}

\noindent\rule[-1ex]{\textwidth}{2pt}\\[4.5ex]

Yo, \textbf{César Hugo Bárzano Cruz}, alumno de la titulación TITULACIÓN de la \textbf{Escuela Técnica Superior
de Ingenierías Informática y de Telecomunicación de la Universidad de Granada}, con DNI 77138361h, autorizo la
ubicación de la siguiente copia de mi Trabajo Fin de Grado en la biblioteca del centro para que pueda ser
consultada por las personas que lo deseen.

\vspace{6cm}

\noindent Fdo: Nombre Apellido1 Apellido2

\vspace{2cm}

\begin{flushright}
Granada a X de mes de 201 .
\end{flushright}


\chapter*{}
\thispagestyle{empty}

\noindent\rule[-1ex]{\textwidth}{2pt}\\[4.5ex]

D. \textbf{Nombre Apellido1 Apellido2 (tutor1)}, Profesor del Área de XXXX del Departamento YYYY de la Universidad de Granada.

\vspace{0.5cm}

D. \textbf{Nombre Apellido1 Apellido2 (tutor2)}, Profesor del Área de XXXX del Departamento YYYY de la Universidad de Granada.


\vspace{0.5cm}

\textbf{Informan:}

\vspace{0.5cm}

Que el presente trabajo, titulado \textit{\textbf{Título del proyecto, Subtítulo del proyecto}},
ha sido realizado bajo su supervisión por \textbf{Nombre Apellido1 Apellido2 (alumno)}, y autorizamos la defensa de dicho trabajo ante el tribunal
que corresponda.

\vspace{0.5cm}

Y para que conste, expiden y firman el presente informe en Granada a X de mes de 201 .

\vspace{1cm}

\textbf{Los directores:}

\vspace{5cm}

\noindent \textbf{Nombre Apellido1 Apellido2 (tutor1) \ \ \ \ \ Nombre Apellido1 Apellido2 (tutor2)}

\chapter*{Agradecimientos}
\thispagestyle{empty}

       \vspace{1cm}


Poner aquí agradecimientos...

